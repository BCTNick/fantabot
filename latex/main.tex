% WRITE THIS IN THE TERMINAL

% pdflatex -interaction=nonstopmode main.tex
% makeglossaries main
% biber main
% pdflatex -interaction=nonstopmode main.tex
% pdflatex -interaction=nonstopmode main.tex

\documentclass[12pt]{article}

% --- Encoding and Input ---
\usepackage[utf8]{inputenc}
\usepackage[T1]{fontenc}
\usepackage[english]{babel}

% --- Math Packages ---
\usepackage{amsmath}
\usepackage{amssymb}
\usepackage{amsfonts}
\usepackage{mathrsfs}
\usepackage{IEEEtrantools} % For aligned formulas

% --- Custom Math Operators ---
\DeclareMathOperator{\logit}{logit}
\DeclareMathOperator*{\argmax}{arg\,max}
\DeclareMathOperator*{\argmin}{arg\,min}

% --- Layout and Page Setup ---
\usepackage[a4paper, left=2.5cm, right=2.5cm, top=2.5cm, bottom=2.5cm]{geometry}
\usepackage[headings]{fullpage}

% --- Table Packages ---
\usepackage{booktabs}         % For professional-looking tables
\usepackage{caption}          % For customizing captions
\usepackage[flushleft]{threeparttable} % For notes below tables

% --- Code Listings ---
\usepackage{listings}
\usepackage{xcolor}
\definecolor{lightblue}{rgb}{0.9, 0.95, 1} % Custom background color for code

\lstset{
  language=R,
  basicstyle=\ttfamily\footnotesize,
  keywordstyle=\color{blue},
  commentstyle=\color{green!50!black},
  stringstyle=\color{red},
  backgroundcolor=\color{lightblue},
  breaklines=true,
  showstringspaces=false,
  numbers=none
}

% --- Graphics and Floating Elements ---
\usepackage{graphicx}
\usepackage{float}           % Improved float placement

% --- Referencing and Hyperlinks ---
\usepackage{hyperref}

% --- Glossary ---
\usepackage[acronym, toc]{glossaries}
\makeglossaries
% Come citare le parole del glossario nel testo:
% \gls{ai} → "Artificial Intelligence (AI)" prima volta, poi "AI"
% \glspl{disposizione} → plurale: "disposizioni"
% \Gls{ml} → con maiuscola iniziale
% \glsfirst{ai} → forza sempre la forma estesa
% \glsshort{ai} → forza solo la forma breve: "AI"
% \glslong{ai} → forza solo la forma estesa: "Artificial Intelligence"

% ========== Regular terms (non-acronyms)  ==========
\newglossaryentry{corpus_normativo}{
    name=corpus normativo,
    description={Insieme completo e organizzato di tutte le norme giuridiche che regolano un determinato ordinamento o una specifica materia}
}

\newglossaryentry{disposizione}{
    name=disposizione,
    plural=disposizioni,
    description={Singolo precetto o comando normativo contenuto all'interno di un atto normativo più ampio (e di cui, quindi, le disposizioni sono le unità minime, spesso ma non necessariamente coincidendo con articoli o commi)}
}

\newglossaryentry{legge}{
    name=legge,
    description={Atto normativo formale emanato dal potere legislativo secondo procedure costituzionalmente stabilite}
}

\newglossaryentry{linguaggio naturale}{
    name=linguaggio naturale,
    description={Sistema di comunicazione usato spontaneamente dagli esseri umani, come l’italiano o l’inglese, ricco e flessibile ma spesso ambiguo.}
}

\newglossaryentry{linguaggio formale}{
    name=linguaggio formale,
    description={Linguaggio artificiale definito da un insieme preciso di simboli e regole, creato per eliminare ambiguità e descrivere concetti in modo rigoroso (es. matematica, logica, programmazione).}
}


% ========== Acronyms ==========
\newacronym{ai}{AI}{Artificial Intelligence}
\newglossaryentry{ai-def}{
    name=Artificial Intelligence,
    description={Sistema tecnologico che simula processi cognitivi umani per risolvere problemi complessi attraverso algoritmi e machine learning},
    see=[Acronym:]{ai}
}

\newacronym{dl}{DL}{Deep Learning}
\newglossaryentry{dl-def}{
    name=Deep Learning,
    description={Sottocampo del machine learning che utilizza reti neurali artificiali con molteplici strati per apprendere rappresentazioni complesse dei dati},
    see=[Acronym:]{dl}
}

\newacronym{llm}{LLM}{Large Language Model}
\newglossaryentry{llm-def}{
    name=Large Language Model,
    description={Modello di intelligenza artificiale basato su reti neurali profonde, addestrato su enormi quantità di testo per comprendere e generare linguaggio naturale. Utilizza architetture transformer e tecniche di apprendimento automatico per elaborare, analizzare e produrre testo in modo contestualmente appropriato},
    see=[Acronym:]{llm}}


\newacronym{ml}{ML}{Machine Learning}
\newglossaryentry{ml-def}{
    name=Machine Learning,
    description={Branca dell'intelligenza artificiale che permette ai computer di apprendere e migliorare automaticamente attraverso l'esperienza senza essere esplicitamente programmati},
    see=[Acronym:]{ml}
}

\newacronym{nlp}{NLP}{Natural Language Processing}
\newglossaryentry{nlp-def}{
    name=Natural Language Processing,
    description={Campo dell'intelligenza artificiale che si occupa dell'interazione tra computer e linguaggio umano, permettendo alle macchine di comprendere, interpretare e generare testo naturale},
    see=[Acronym:]{nlp}
} % Importa le definizioni dal file separato
\renewcommand*{\glssymbol}[1]{\space(#1)}

% --- Bibliography ---
\usepackage{csquotes}
\usepackage[
    backend=biber,
    style=apa,
    maxcitenames=2,
    mincitenames=1,
    sorting=ynt
]{biblatex}
\addbibresource{bibfile.bib}

% --- Page Headers and Footers ---
\usepackage{fancyhdr}
\setlength{\headheight}{15pt}
\pagestyle{fancy}
\fancyhf{}
\rhead{Scognamiglio Nicola}
\rfoot{Page \thepage}

% --- Miscellaneous ---
\usepackage{comment}



\begin{document}

\begin{titlepage}
        \centering % Center all text
        \vspace*{\baselineskip} % White space at the top of the page
        
        {\huge Policy-Value Network for Budget Constrained Sequential Multi-Item English Auction}\\[0.2\baselineskip] % Title
        
        
        \vspace*{\baselineskip}
        
        {\Large --- Machine Learning Project ---\\}

        \vspace*{\baselineskip}
        
        {\LARGE Scognamiglio Nicola\\[\baselineskip]} % Editor list  
        
        \vspace*{\baselineskip}

        \vfill
        
        Napoli, 2025 \par % Location and year
        
        \vspace*{\baselineskip}

        {\itshape Università degli Studi di Napoli - Federico II\par} % Editor affiliation
    \end{titlepage}


\tableofcontents
\clearpage


\begin{abstract}
kkkkk
\end{abstract}
\clearpage


\section{Introduction}

In fantasy football leagues, player selection often occurs through an auction mechanism in which participants bid for multiple players sequentially while managing a limited budget. 
This setting can be formally modeled as a \textit{budget-constrained sequential multi-item English auction}, where each agent must strategically allocate resources across rounds, balancing between immediate opportunities and long-term goals.

Traditional auction theory provides analytical insights into single-item or simultaneous multi-item settings, but real-world fantasy football drafts introduce additional complexity. 
Agents must reason dynamically under uncertainty, learning optimal bidding policies as new information (such as opponents’ remaining budgets or players’ availability) unfolds.

Recent advances in artificial intelligence, particularly in reinforcement learning, have demonstrated how \textit{policy--value networks} can learn to approximate optimal strategies in complex sequential decision problems. 
Inspired by models such as AlphaZero, this work proposes to apply a policy--value approach to auction environments, enabling agents to make data-driven and adaptive bidding decisions.

The objective of this study is to explore how a policy--value network can be designed, trained, and evaluated within the fantasy football auction framework. 
The research combines concepts from auction theory, machine learning, and game theory to provide a computational perspective on strategic resource allocation under constraints.

\section{Background and Related Work}

This section will review the theoretical foundations of English auctions, budget constraints, and sequential decision-making, as well as related work in deep reinforcement learning and auction design.

\section{Modeling the Auction Environment}

Describe the auction setup, bidding rules, and constraints. Define the state space, actions, and reward functions.

\section{Policy--Value Network Architecture}

Explain the neural network design, input representation, and training algorithm.

\section{Experiments and Results}

Summarize simulation setup, performance metrics, and comparison with baseline bidding strategies.

\section{Discussion and Future Work}

Discuss findings, limitations, and possible extensions such as incorporating opponent modeling or real fantasy football data.

\section{Conclusion}

Conclude with the main insights on how policy--value methods can enhance bidding efficiency and fairness in sequential auction environments.

% Stampa il glossario 
\clearpage
\glsaddall% Mostra tutti i termini definiti anche se non usati
\printglossary[type=main, title=Glossario, nonumberlist]
% Stampa il glossario degli acronimi
\printglossary[type=\acronymtype, title=Acronimi, nonumberlist]

% Stampa bibliografia
\newpage
\nocite{*}
\printbibliography[title=Bibliografia, heading=bibintoc]
\end{document}